\documentclass[specification,annotation,times]{itmo-student-thesis}

%% Опции пакета:
%% - specification - если есть, генерируется задание, иначе не генерируется
%% - annotation - если есть, генерируется аннотация, иначе не генерируется
%% - times - делает все шрифтом Times New Roman, собирается с помощью xelatex
%% - languages={...} - устанавливает перечень используемых языков. По умолчанию это {english,russian}.
%%                     Последний из языков определяет текст основного документа.

%% Делает запятую в формулах более интеллектуальной, например:
%% $1,5x$ будет читаться как полтора икса, а не один запятая пять иксов.
%% Однако если написать $1, 5x$, то все будет как прежде.
\usepackage{icomma}

%% Один из пакетов, позволяющий делать таблицы на всю ширину текста.
\usepackage{tabularx}

%% Данные пакеты необязательны к использованию в бакалаврских/магистерских
%% Они нужны для иллюстративных целей
%% Начало
\usepackage{tikz}
\usetikzlibrary{arrows}
\usepackage{filecontents}
\begin{filecontents}{master-thesis.bib}
@incollection{ nsga-ii-steady-state,
    year        = {2009},
    booktitle   = {Nature-Inspired Algorithms for Optimisation},
    number      = {193},
    series      = {Studies in Computational Intelligence},
    title       = {On the Effect of Applying a Steady-State Selection Scheme in the Multi-Objective Genetic Algorithm {NSGA}-{II}},
    publisher   = {Springer Berlin Heidelberg},
    author      = {Nebro, Antonio J. and Durillo, Juan J.},
    pages       = {435-456},
    langid      = {english}
}


@inproceedings{ example-english,
    year        = {2015},
    booktitle   = {Proceedings of IEEE Congress on Evolutionary Computation},
    author      = {Maxim Buzdalov and Anatoly Shalyto},
    title       = {Hard Test Generation for Augmenting Path Maximum Flow 
                   Algorithms using Genetic Algorithms: Revisited},
    pages       = {2121-2128},
    langid      = {english}
}

@article{ example-russian,
    author      = {Максим Викторович Буздалов},
    title       = {Генерация или нет тестов для олимпиадных задач по программированию 
                   с использованием генетических алгоритмов},
    journal     = {Научно-технический вестник {СПбГУ} {ИТМО}},
    number      = {2(72)},
    year        = {2011},
    pages       = {72-77},
    langid      = {russian}
}

@article{ unrestricted-jump-evco,
    author      = {Maxim Buzdalov and Benjamin Doerr and Mikhail Kever},
    title       = {The Unrestricted Black-Box Complexity of Jump Functions},
    journal     = {Evolutionary Computation},
    year        = {2016},
    note        = {Accepted for publication},
    langid      = {english}
}

@book{ bellman,
    author      = {R. E. Bellman},
    title       = {Dynamic Programming},
    address     = {Princeton, NJ},
    publisher   = {Princeton University Press},
    numpages    = {342},
    pagetotal   = {342},
    year        = {1957},
    langid      = {english}
}
\end{filecontents}
%% Конец

%% Указываем файл с библиографией.
\addbibresource{master-thesis.bib}

%% Добавляем подсветку синтаксиса для ECMAScript 6
\input{js_styles}

\begin{document}

\studygroup{M4239}
\title{Пример оформления ВКР магистра}
\author{Буздалов Максим Викторович}{Буздалов М.В.}
\supervisor{Шалыто Анатолий Абрамович}{Шалыто А.А.}{проф., д.т.н.}{главный научный сотрудник Университета ИТМО}
\publishyear{2019}
%% Дата выдачи задания. Можно не указывать, тогда надо будет заполнить от руки.
\startdate{01}{сентября}{2017}
%% Срок сдачи студентом работы. Можно не указывать, тогда надо будет заполнить от руки.
\finishdate{31}{мая}{2019}
%% Дата защиты. Можно не указывать, тогда надо будет заполнить от руки.
\defencedate{15}{июня}{2019}

\addconsultant{Белашенков Н.Р.}{канд. физ.-мат. наук, без звания}
\addconsultant{Беззубик В.В.}{без степени, без звания}

\secretary{Павлова О.Н.}

%% Задание
%%% Техническое задание и исходные данные к работе
\technicalspec{Требуется разработать стилевой файл для системы \LaTeX, позволяющий оформлять бакалаврские работы и магистерские диссертации
на кафедре компьютерных технологий Университета ИТМО. Стилевой файл должен генерировать титульную страницу пояснительной записки,
задание, аннотацию и содержательную часть пояснительной записк. Первые три документа должны максимально близко соответствовать шаблонам документов,
принятым в настоящий момент на кафедре, в то время как содержательная часть должна максимально близко соответствовать ГОСТ~7.0.11-2011
на диссертацию.}

%%% Содержание выпускной квалификационной работы (перечень подлежащих разработке вопросов)
\plannedcontents{Пояснительная записка должна демонстрировать использование наиболее типичных конструкций, возникающих при составлении
пояснительной записки (перечисления, рисунки, таблицы, листинги, псевдокод), при этом должна быть составлена так, что демонстрируется
корректность работы стилевого файла. В частности, записка должна содержать не менее двух приложений (для демонстрации нумерации рисунков и таблиц
по приложениям согласно ГОСТ) и не менее десяти элементов нумерованного перечисления первого уровня вложенности (для демонстрации корректности
используемого при нумерации набора русских букв).}

%%% Исходные материалы и пособия 
\plannedsources{\begin{enumerate}
    \item ГОСТ~7.0.11-2011 <<Диссертация и автореферат диссертации>>;
    \item С.М. Львовский. Набор и верстка в системе \LaTeX;
    \item предыдущий комплект стилевых файлов, использовавшийся на кафедре компьютерных технологий.
\end{enumerate}}

%%% Цель исследования
\researchaim{Разработка удобного стилевого файла \LaTeX
             для бакалавров и магистров кафедры компьютерных технологий.}

%%% Задачи, решаемые в ВКР
\researchtargets{\begin{enumerate}
    \item обеспечение соответствия титульной страницы, задания и аннотации шаблонам, принятым в настоящее время на кафедре;
    \item обеспечение соответствия содержательной части пояснительной записки требованиям ГОСТ~7.0.11-2011 <<Диссертация и автореферат диссертации>>;
    \item обеспечение относительного удобства в использовании~--- указание данных об авторе и научном руководителе один раз и в одном месте, автоматический подсчет числа тех или иных источников.
\end{enumerate}}

%%% Использование современных пакетов компьютерных программ и технологий
\addadvancedsoftware{Пакет \texttt{tabularx} для чуть более продвинутых таблиц}{\ref{sec:tables}, Приложения~\ref{sec:app:1}, \ref{sec:app:2}}
\addadvancedsoftware{Пакет \texttt{biblatex} и программное средство \texttt{biber}}{Список использованных источников}

%%% Краткая характеристика полученных результатов 
\researchsummary{Получился, надо сказать, практически неплохой стилевик. В 2015--2018 годах
его уже использовали некоторые бакалавры и магистры. Надеюсь на продолжение.}

%%% Гранты, полученные при выполнении работы 
\researchfunding{Автор разрабатывал этот стилевик исключительно за свой счет и на
добровольных началах. Однако значительная его часть была бы невозможна, если бы
автор не написал в свое время кандидатскую диссертацию в \LaTeX,
а также не отвечал за формирование кучи научно-технических отчетов по гранту,
известному как <<5-в-100>>, что происходило при государственной финансовой поддержке
ведущих университетов Российской Федерации (субсидия 074-U01).}

%%% Наличие публикаций и выступлений на конференциях по теме выпускной работы
\researchpublications{По теме этой работы я (к счастью!) ничего не публиковал.
\begin{refsection}
Однако покажу, как можно ссылаться на свои публикации из списка литературы:
\nocite{example-english, example-russian}
\printannobibliography
\end{refsection}
}

%% Эта команда генерирует титульный лист и аннотацию.
\maketitle{Магистр}

%% Оглавление
\tableofcontents





%% Макрос для введения. Совместим со старым стилевиком.




\startprefacepage

{\large \textbf{Актуальность работы}}
\bigbreak

{\large О компании Fulldive}
\\
Успех VR приложения и, как следствие, появление идеи развития браузера
\bigbreak

{\large Идеология браузера:}
\begin{itemize}
	\item убирать всю рекламу на сайтах
	\item вместо нее показывать собственную рекламу
	\item прибыль полученную за публикацию своей рекламы делить с пользователями пополам
\end{itemize}
\bigbreak

{\large Плюсы:}
\begin{itemize}
	\item "органическое" привлечение пользователей, желающих монетизировать свое времяпрепровождении в интернете
	\item у наших пользователей появляется возможность монетизировать свое время, проведенное в интернете (данный вид заработка особенно актуален в странах с низким уровнем доходов и занятости населения. Молодые люди, проходящие обучение в средних или высших учебных заведениях, так или иначе использующие интернет в своих образовательных и развлекательных целях, теперь могут еще и получать за это деньги. Более того, подобный способ получение дохода как никогда актуален в текущий период мирового экономического кризиса, массового сокращения рабочих мест и увеличение доли малообеспеченного населения).
\end{itemize}
\bigbreak

{\large Минусы:}
\begin{itemize}
	\item крупные рекламные площадки либо не одобряют такую схему разделения доходов, либо пока относятся к этому индифферентно
\end{itemize}
\bigbreak

\bigbreak
{\large \textbf{Цель работы}}
\bigbreak

Целью данной работы является создание собственной рекламной сети \textbf{UniAds}, для достижения которой необходимо разработать соответствующее программное обеспечение.
\bigbreak

\bigbreak
{\large \textbf{Практическое значение работы}} (причины создания собственной рекламной сети)
\bigbreak

\begin{enumerate}
	\item Диверсификация риска сокращения прибыли, получаемой компанией за осуществление показов рекламы, из-за прекращения сотрудничества с основными рекламными сетями
	\item Индивидуальной конфигурация форматов рекламных материалов
	\item Контроль рекламного инвентаря в мобильных приложениях компании
	\item Сбор данных первого порядка (First-Party Data). Информация о пользователях, собранная владельцами веб-сайта. Таковая представляет наивысшую ценность, так как является наиболее точной и качественной и принадлежит компании.
	\item Возможность привлечения сторонних рекламодателей в различных странах
\end{enumerate}
\bigbreak

(большая аудитория VR приложения (в каких странах пользуются?) -> (гипотеза) ожидается рост аудитории браузера (желательно сроки и цифры, хотя бы примерно) в таких-то странах (список на основе статистики использования Browser / VR (так как можно перелить часть аудитории в браузер)) -> (гипотеза) в этих странах локальные рекламодатели, которые захотят получить доступ к нашей аудитории? (нужна ли она им? скорее всего да, если довольно большая) -> (гипотеза) используют нашу рекламную сеть (какие есть конкуренты? почему они должны выбрать именно нас?)
\bigbreak

\bigbreak
{\large \textbf{Краткое описание структуры работы}}
\bigbreak

В Главе X рассмотрен вопрос…




%% Начало содержательной части.




\chapter{Обзор предметной области}


\startrelatedwork %% Так помечается начало обзора.

\begin{quotation}
  Описание того как на данный момент устроены рекламные платформы, термины и общая структура
\end{quotation}

В данной предметной области достаточно много различных терминов, поэтому 
определим их прежде чем сформулируем подробную цель работы.

\section{Используемые термины и понятия}\label{sec:terms}

%%

\textbf{Рекламная сеть (Ad Network)}
\\
An advertising buying network is an online platform that acts as an intermediary between advertisers and publishers. It collects the publisher's inventory and matches it with demand coming from advertisers. Ad networks are able to sort inventory into categories according to the specific audience segments.
\\
Целевая рекламная сеть (targeted ad network) работает точно. Наиболее продвинутый вид сетей, в связи с чем их также называют сетями 2.0. Такие сети специализируются на технологиях точного таргетирования по поведенческим или контекстным признакам, активно анализируют данные о пользователях в целях повышения стоимости инвентаря, который они продают
\\
Дополнительно описание берем отсюда https://smartyads.com/blog/create-your-own-ad-network/
\\
Примеры рекламных сетей: Facebook, Google AdMob, ...

%%

\textbf{Основные роли в рекламной сети}
\\
Рекламодатель (Advertiser)
\\
Рекламодатель как пользователь рекламной сети:
создает рекламные материалы различных форматов (ad format) = "типов инвентаря" (inventory type) рекламной сети
конфигурирует (таргетирует) аудиторию для данного рекламного материала
указывает белый / черный список возможных плейсментов (отдельные площадки или категории площадок, отдельные разделы конкретных площадок, …)
\\
Издатель (Publisher)
\\
Publisher как пользователь рекламной сети:
создает (регистрирует) плейсменты доступных "типов инвентаря" рекламной сети, указывает параметры площадки: тематику, тип и объем доступной ему аудитории и т.д.
указывает белый / черный список возможных рекламодателей / рекламных агенств (отдельные или по категориям)
\\
Advertisers don't know which websites serve their ads. Publishers don't know what companies buy their inventory.

%%

\textbf{DSP / SSP, Ad server / Publisher server}
\\
... тут описания понятий и схемы технического устройства

%%

\textbf{Таргетирование}
Решение о показе того или иного рекламного объявления принимается на основе следующей информации:
\begin{itemize}
	\item контекста плейсмента (inventory placement) для рекламы (Content and contextual targeting - \url{https://en.wikipedia.org/wiki/Contextual_advertising}),
	\item накопленных данных о пользователе (Behavioral targeting - \url{https://en.wikipedia.org/wiki/Targeted_advertising#Behavioural_targeting}): страна, соц-дем, устройства, категории интереса (подробнее это все расписано в TargetingOptions) (Amazon <-> IBM Watson category mapper)
\end{itemize}
\bigbreak

Targeting Options \url{https://adprofs.co/how-to-evaluate-demand-side-platform-companies/#targeting}

%%

\textbf{Трекинг}
Tracking data: Track / Impression / Click / Sell
\\
Основная воронка трекинга, open vs closed funnels

%%

\begin{quotation}
  Сайт = рекламная площадка (в нашем случае это мобильное приложение)
  \\
  OpenRate = fill rate

\end{quotation}

%%

\section{Основной бизнес-процесс}

стадия 0: поиск и интеграция исходных рекламодателей
\\
результат 0: исходные рекламодатели (Amazon, EBay, Aliexpress, ...)
\\
стадия 1: фильтрация товаров и обогащение сторонними данными (отзывы, локализация, …)
\\
результат 1: рекламные объявления UniAds или рекламные провайдеры: UniAds, Facebook, Google AdMob, …
\\
стадия 2: фильтрация и ранжирование уже созданных объявлений
\\
результат 2: показы объявлений
\\
стадия 3:
\\
результат 3: переходы (клики)
\\
стадия 4:
\\
результат 4: конверсии (продажи, установки приложения, …)
\\
стадия 5: выделение конверсий с максимальной возможной прибылью
\\
результат 5: прибыль
\\	

\finishrelatedwork %% Так помечается конец обзора.


\section{Постановка задачи}


Для достижения поставленной цели необходимо решить основную задачу, которая заключается в разработке программного обеспечения, необходимого для полноценного функционирования рекламной сети.

Для удобства и простоты дальнейшего описания подзадач, под аббревиатурой ПО будем понимать программное обеспечение, необходимое для решение основной задачи.

Решение основной задачи согласно выбранной (каскадной) модели разработки состоит из следующих стадий:
\begin{itemize}
	\item сформировать требования к ПО
	\item ? провести анализ предметной области на основе разработанных требований: существуют ли готовые решения или отдельные компоненты
	\item спроектировать архитектуру ПО
	\item реализовать ПО на основе требований и модели архитектуры
	\item протестировать итоговое ПО
	\item внедрить ПО
\end{itemize}
\bigbreak


\section{Требования}\label{sec:requirements}


\textbf{Общие базовые (Бизнес-требования)}
\begin{itemize}
	\item Возможность показывать товары с Amazon в виде рекламных объявлений
	\item Компания Fulldive единственный Publisher и единственный Advertiser (в будущем могут быть и другие рекламодатели)
	\item Модели ценообразования CPC и CPM (другие не поддерживаем)
	\item возможность создания и получения единичных объявлений
	\item возможность получения списка рекламных объявлений для показа на одной странице
	\item глобальный Postback для учета конверсий: покупок, установок приложения и т.д. (нужен только для модели CPA: CPI, CPS, …)
	\item трекинг данных статистики показов, переходов и пр. каждого рекламного объявления
	\item сбор данных по данным трекинга
\end{itemize}

\textbf{Требования к показу единичных объявлений:}
\begin{itemize}
	\item возможность задания категории объявления
	\item возможность задания списка стран, в которых будет показано объявление
\end{itemize}

\textbf{Требования к показу списка объявлений:}
\begin{itemize}
	\item консистентность списка для пользователя / конкретного девайса пользователя
	\item возможность фильтрации провайдеров / товаров?
	\item возможность учета для каждого товара из списка: показов и кликов
\end{itemize}

\textbf{Требования к форматам рекламных объявлений}
\begin{itemize}
	\item поддержка нативного формата (изображения с различными разрешениями + текст)
	\item поддержка баннерного формата (HTML разметка)
	\item поддержка видеоформата
\end{itemize}

\textbf{Требования к аналитике:}
\begin{itemize}
	\item возможность просмотра статистики каждого рекламного объявления в формате (AdId, Показы, Клики, CTR, тут колонки таблицы) с фильтрацией по … и группировкой по …
	\item возможность просмотра параметров эффективности (OR, CTR) стратегий таргетинга, рассчитанных по данным трекинга, с фильтрацией по интервалу дат и группировкой по идентификатору стратегии
\end{itemize}

\textbf{Технические требования:} количественные параметры (количество показов, кликов в секунду / сутки, максимальное количество ошибок за сутки / месяц)

\textbf{Общие сценарии использования}
\begin{itemize}
	\item Издатель запрашивает одно рекламное объявление
	\item Издатель запрашивает список рекламных объявлений
\end{itemize}

\textbf{Сценарии использования накопленных данных трекинга (ответ на вопрос: для чего мы собираем эти данные?):}
\begin{itemize}
	\item (ручное использование) расчет (real-time?) статистики по рекламным объявлениям: Impressions, Clicks, Conversions? (в основном для сторонних рекламодателей), CTR, CR для отображение в панели рекламодателя
	\item (автоматическое использование) использование CTR рекламного объявления в стратегиях таргетирования (Global Interesting Strategy, Interesting in Category Strategy, …)
	\item (ручное использование) оптимизация бизнес процесса на основе данных (Data-driven decision making)
	\item  (автоматическое использование) автоматический выбор стратегии таргетирования на основе статистики стратегий
\end{itemize}


\chapterconclusion


\chapter{Теоретическое решение}

В данной главе описываются основные компоненты рекламной сети, клиент-серверное взаимодействие, ...

Исходя из требований и сценариев использования приведенных в \ref{sec:requirements} было принято решение выбрать трёхуровневую клиент-серверную архитектуру, в которой:
\begin{itemize}
	\item Cервер (Ad Server - \url{https://en.wikipedia.org/wiki/Ad_serving}) отвечает за управление рекламным инвентарем и рекламными объявлениями, сбор статистики показов, переходов и конверсий. Состоит из связующего слоя (приложения) и слоя данных (базы данных).
	\item Клиент отвечает за корректное отображение полученных от сервера рекламных материалов.
\end{itemize}

\section{Клиент-серверное взаимодействие}

Взаимодействие клиента и сервера происходит по заранее установленному контракту, реализованному поверх HTTP протокола и представляет из себя несколько сценариев, которые будут подробнее описаны далее.

\subsection{Получение рекламного объявления}
Получение данных пользователя. 
Выбор стратегии таргетинга. 
Получение списка объявлений по выбранной стратегии.

\subsection{Получение списка рекламных объявлений}
Получение данных пользователя и текущей сессии (у каждой сессии свой уникальный ключ). 
Выбор стратегии таргетинга. 
Получение списка объявлений по выбранной стратегии.

\subsection{Трекинг показов и переходов}
...

\subsection{Получение статистики по стратегиям и рекламных объявлениям}
...

\section{Таргетирование}


Для таргетирования используются 4 стратегии:
\begin{enumerate}
	\item случайные объявления
	\item объявления с глобально высоким показателем CTR
	\item объявления с высоким показателем CTR в категориях интереса данного пользователя, рассчитанных на основе просмотренных страниц в интернете
	\item объявления с высоким показателем CTR в категориях интереса данного пользователя, рассчитанных на основе кликов по объявлениям внутри приложения
\end{enumerate}

Каждая из приведенных стратегия таргетирования назначает веса рекламным объявлениям \url{https://en.wikipedia.org/wiki/Creative_sequencing}, чтобы управлять их частотой показа. Также применяется рандомизация для равномерного показа объявлений внутри своей группы (подтвержденные, отклоненные, тестируемые)

Выбор стратегии на основе данных, представленных в запросе:

Если рекламное объявление должно зависеть от контекста, в котором оно будет показано выбирается одна из стратегий, которая таргетирует на основании параметров переданного контекста. Иначе выбирается стратегия, которая таргетирует на основании накопленных знаниях о категориях интереса текущего пользователя.

\section{Актуализация рекламных материалов}

Рекламодатель создает рекламные материалы на основе товаров из Amazon, сохраняет их в базе данных и обновляет список актуальных рекламных материалов, доступных для получения клиентами.

(Схема)


\section{Актуализация рекламного инвентаря}

Издатель запрашивает актуальный рекламный инвентарь на сервере конфигурации и сохраняет его в базе данных.

(Схема)

\chapterconclusion

Вывод в конце второй главы

\chapter{Практическое решение}

Решено было реализовать ПО в виде JSON API

Здесь описывается итоговая архитектура рекламной сети (инфраструктура, количество серверов, тип масштабирования, базы данных, кэширование...

Техническая подзадача 1: выбор оптимальной стратегии таргетирования (A/B тестирование стратегий)
как решать 
почему именно так решать

Техническая подзадача 2: StrategyState resolving (N.B. sessionKey по идее может быть произвольной строкой предоставляемой Publisher'ом)

Техническая подзадача 3: StrategyState threads (обновление состояний потока по расписанию (а также при запросе?))

Техническая подзадача 4: таргетирование = ранжирование (относительные величины - порядковые номера)/ скоринг (абсолютные величины)+ эвристики фильтрации + небольшая рандомизация (в будущем можно сделать на основе TargetingOptions (для этого можно использовать сторонее решение? какое?)

Техническая подзадача 5: трекинговая закрытая воронка (Track / Impression / Click / Sell)

Техническая подзадача 6: актуальность кэша рекламных объявлений


\chapterconclusion

Вывод в конце третьей главы

%% Макрос для заключения. Совместим со старым стилевиком.
\startconclusionpage

Дальнейшее развитие (стратегические направления / цели) (исходя из них можно уточнить требования к реализации платформы 🤔 )
(все дальнейшие разработки и улучшения также направлены на увеличение прибыли (согласно основному бизнес-процессу итоговая цель - прибыль))
№1 Развитие компании в качестве своего собственного рекламодателя (growth as Advertiser / Ad provider)
Как это сделать?
Идея №1: увеличить количество исходных рекламодателей - интегрировать E-commerce платформы Ebay, Aliexpress, Wildberries, …
Идея №2: подбирать множество товаров таким образом, чтобы итоговое число покупок было максимально (слишком мало товаров - неудовлетворенный спрос, слишком много товаров - проблема выбора и рассеивание внимания)
Идея №3: тиражирование товаров с максимально высокой вероятностью покупки для конкретного пользователя, например, товары из интересующих пользователя категорий (персонализация показываемой рекламы)
№2 Оптимизация количества и качества плейсментов внтури приложений компании (growth as Publisher)
№3 Привлечение сторонних рекламодателей (добавление RTB / Bidders)

\printmainbibliography

%% После этой команды chapter будет генерировать приложения, нумерованные русскими буквами.
%% \startappendices из старого стилевика будет делать то же самое
\appendix

\chapter{Пример приложения}\label{sec:app:1}
                

\end{document}