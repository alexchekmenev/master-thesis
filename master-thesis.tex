\documentclass[times]{itmo-student-thesis}

%% Опции пакета:
%% - specification - если есть, генерируется задание, иначе не генерируется
%% - annotation - если есть, генерируется аннотация, иначе не генерируется
%% - times - делает все шрифтом Times New Roman, собирается с помощью xelatex
%% - languages={...} - устанавливает перечень используемых языков. По умолчанию это {english,russian}.
%%                     Последний из языков определяет текст основного документа.

%% Делает запятую в формулах более интеллектуальной, например:
%% $1,5x$ будет читаться как полтора икса, а не один запятая пять иксов.
%% Однако если написать $1, 5x$, то все будет как прежде.
\usepackage{icomma}

%% Один из пакетов, позволяющий делать таблицы на всю ширину текста.
\usepackage{tabularx}

%% Данные пакеты необязательны к использованию в бакалаврских/магистерских
%% Они нужны для иллюстративных целей
%% Начало
\usepackage{tikz}
\usetikzlibrary{arrows}
\usepackage{filecontents}
\begin{filecontents}{master-thesis.bib}

@article{ example-russian,
    author      = {Максим Викторович Буздалов},
    title       = {Генерация тестов для олимпиадных задач по программированию 
                   с использованием генетических алгоритмов},
    journal     = {Научно-технический вестник {СПбГУ} {ИТМО}},
    number      = {2(72)},
    year        = {2011},
    pages       = {72-77},
    langid      = {russian}
}

@online{ programmatic-buying,
    title       = {Programmatic Buying - Definition and Demarcation},
    url         = {https://en.ryte.com/wiki/Programmatic_Buying},
    langid      = {english}
}

\end{filecontents}
%% Конец

%% Указываем файл с библиографией.
\addbibresource{master-thesis.bib}

%% Добавляем подсветку синтаксиса для ECMAScript 6
\usepackage{listings}
\usepackage{color}
\usepackage{textcomp} % for upquote

%
% ECMAScript 2015 (ES6) definition by Gary Hammock
%

\lstdefinelanguage[ECMAScript2015]{JavaScript}[]{JavaScript}{
  morekeywords=[1]{await, async, case, catch, class, const, default, do,
    enum, export, extends, finally, from, implements, import, instanceof,
    let, static, super, switch, throw, try, public, constructor},
  morestring=[b]` % Interpolation strings.
}


%
% JavaScript version 1.1 by Gary Hammock
%
% Reference:
%   B. Eich and C. Rand Mckinney, "JavaScript Language Specification
%     (Preliminary Draft)", JavaScript 1.1.  1996-11-18.  [Online]
%     http://hepunx.rl.ac.uk/~adye/jsspec11/titlepg2.htm
%

\lstdefinelanguage{JavaScript}{
  morekeywords=[1]{break, continue, delete, else, for, function, if, in,
    new, return, this, typeof, var, void, while, with},
  % Literals, primitive types, and reference types.
  morekeywords=[2]{false, null, true, boolean, number, undefined,
    Array, Boolean, Date, Math, Number, String, Object},
  % Built-ins.
  morekeywords=[3]{eval, parseInt, parseFloat, escape, unescape},
  sensitive,
  morecomment=[s]{/*}{*/},
  morecomment=[l]//,
  morecomment=[s]{/**}{*/}, % JavaDoc style comments
  morestring=[b]',
  morestring=[b]"
}[keywords, comments, strings]


\lstalias[]{ES6}[ECMAScript2015]{JavaScript}

% Requires package: color.
\definecolor{mediumgray}{rgb}{0.3, 0.4, 0.4}
\definecolor{mediumblue}{rgb}{0.0, 0.0, 0.8}
\definecolor{forestgreen}{rgb}{0.13, 0.55, 0.13}
\definecolor{darkviolet}{rgb}{0.58, 0.0, 0.83}
\definecolor{royalblue}{rgb}{0.25, 0.41, 0.88}
\definecolor{crimson}{rgb}{0.86, 0.8, 0.24}

\lstdefinestyle{JSES6Base}{
  backgroundcolor=\color{white},
  basicstyle=\ttfamily,
  breakatwhitespace=false,
  breaklines=false,
  captionpos=b,
  columns=fullflexible,
  commentstyle=\color{mediumgray}\upshape,
  emph={},
  emphstyle=\color{crimson},
  extendedchars=true,  % requires inputenc
  fontadjust=true,
  frame=single,
  identifierstyle=\color{black},
  keepspaces=true,
  keywordstyle=\color{mediumblue},
  keywordstyle={[2]\color{darkviolet}},
  keywordstyle={[3]\color{royalblue}},
  numbers=left,
  numbersep=5pt,
  numberstyle=\tiny\color{black},
  rulecolor=\color{black},
  showlines=true,
  showspaces=false,
  showstringspaces=false,
  showtabs=false,
  stringstyle=\color{forestgreen},
  tabsize=2,
  title=\lstname,
  upquote=true  % requires textcomp
}

\lstdefinestyle{JavaScript}{
  language=JavaScript,
  style=JSES6Base
}
\lstdefinestyle{ES6}{
  language=ES6,
  style=JSES6Base
}

\begin{document}

\studygroup{M42381}
\title{Разработка платформы для управления
таргетированной рекламой}
\author{Чекменев Александр Романович}{Чекменев А.Р.}
\supervisor{Фильченков Андрей Александович}{Фильченков А. А.}{к.ф.-м.н.}{доц. ФИТиП}
\publishyear{2020}
%% Дата выдачи задания. Можно не указывать, тогда надо будет заполнить от руки.
\startdate{01}{сентября}{2019}
%% Срок сдачи студентом работы. Можно не указывать, тогда надо будет заполнить от руки.
%\finishdate{31}{мая}{2020}
%% Дата защиты. Можно не указывать, тогда надо будет заполнить от руки.
%\defencedate{15}{июня}{2019}

%\addconsultant{Белашенков Н.Р.}{канд. физ.-мат. наук, без звания}
%\addconsultant{Беззубик В.В.}{без степени, без звания}

\secretary{Павлова О.Н.}

%% Задание
%%% Техническое задание и исходные данные к работе
\technicalspec{Требуется разработать программное обеспечение, обеспечивающее работу требуемых функций рекламной сети согласно установленным требованиям.}

%%% Содержание выпускной квалификационной работы (перечень подлежащих разработке вопросов)
\plannedcontents{}

%%% Исходные материалы и пособия 
\plannedsources{\begin{enumerate}
    \item 
\end{enumerate}}

%%% Цель исследования
\researchaim{}

%%% Задачи, решаемые в ВКР
\researchtargets{\begin{enumerate}
    \item разработка таргетированной рекламной сети
\end{enumerate}}

%%% Использование современных пакетов компьютерных программ и технологий

%%% Краткая характеристика полученных результатов 
\researchsummary{Рекламная сеть запущена в production-окружении платформы Google Kubernetes Cloud.Примеры рекламных объявлений различного формата доступны в приложении Fulldive Browser на Android.}

%%% Гранты, полученные при выполнении работы 
\researchfunding{}

%%% Наличие публикаций и выступлений на конференциях по теме выпускной работы
\researchpublications{}

%% Эта команда генерирует титульный лист и аннотацию.
\maketitle{Магистр}

%% Оглавление
\tableofcontents





%% Макрос для введения. Совместим со старым стилевиком.




\startprefacepage

Непрерывное развитие технологий и интеллектуальных систем неизбежно влечет за собой активное развитие всех смежных областей, а также внедрение этих достижений в другие сферы. 
Сфера интернет-рекламы в этом плане не является исключением. 

Результатом применения современных технологий в рекламе является алгоритмическая закупка рекламы (англ. programmatic buying) \cite{programmatic-buying} - способ закупки целевого трафика с оплатой за совершенные целевые действия

Доля интернет-рекламы растет …(ссылка). Основными трендами, которой являются (Персонализация, видео, интерактив, …). Персонализация красной нитью проходит сквозь большинство современных трендов. Для эффективной персонализации в большинстве случаев необходим не менее эффективный таргетинг

В данный момент таргетированная реклама одна из наиболее развивающихся сфер интернет-рекламы, так как позволяет максимально гибко и точно задать необходимую целевую аудиторию для каждого рекламного объявления. Существует ряд компаний, предоставляющих услуги по размещению таргетированной рекламы в своих рекламных сетях. Но все они имеют такие существенные недостатки как отсутствие прямого доступа рекламодателя к целевой аудитории, а также полная зависимость рекламодателя от условий, выдвигаемых рекламной сетью. Все это создает неудобства для компаний, предпочитающих индивидуальный и нестандартный подход к проведению своих рекламных кампаний. Поэтому наличие собственной рекламной сети является неоспоримым преимуществом для компаний имеющих достаточно большой поток активных пользователей.

Компания Fulldive разработала приложение Fulldive VR, которое входит в топ-5 приложений для Android в категории VR. Fulldive Browser - второй и активно развивающийся продукт компании. Одной из его наиболее важных и эффективно влияющих на привлечение новых пользователей возможностью является возможность пользователя монетизировать свое время проведенное за чтением новостей, общением в социальных сетях, просмотром роликов и другими действиями типичными для любого другого браузера. Достигается это за счет того, что компания отдает пользователю часть прибыли, полученной от рекламных сетей за просмотры рекламы, совершенные данным пользователем в нашем приложении. 

Одним из способов увеличения данной прибыли является использование собственной рекламной сети для тиражирования рекламных объявлений пользователям своих приложений, что позволит не отдавать сторонней рекламной сети часть полученных от рекламодателей средств.

Практической пользой собственной рекламной сети для компании является:

\begin{enumerate}
	\item Диверсификация риска сокращения прибыли, получаемой компанией за осуществление показов рекламы, из-за прекращения сотрудничества с основными рекламными сетями
	\item Возможность добавления собственных рекламных форматов
	\item Контроль рекламного инвентаря в мобильных приложениях компании
	\item Сбор данных непосредственного взаимодействия пользователей с рекламой (First-Party Data), которые представляет наивысшую ценность, так как является наиболее точными, качественными и принадлежат компании
	\item Возможность привлечения сторонних рекламодателей в различных странах
\end{enumerate}
\bigbreak


Целью данной работы является создание собственной рекламной сети \textbf{UniAds}, для достижения которой необходимо разработать соответствующее программное обеспечение.

%{\large О компании Fulldive}
%\\
%Успех VR приложения и, как следствие, появление идеи развития браузера
%\bigbreak

%{\large Идеология браузера:}
%\begin{itemize}
%	\item убирать всю рекламу на сайтах
%	\item вместо нее показывать собственную рекламу
%	\item прибыль полученную за публикацию своей рекламы делить с пользователями пополам
%\end{itemize}
%\bigbreak

%{\large Плюсы:}
%\begin{itemize}
%	\item "органическое" привлечение пользователей, желающих монетизировать свое времяпрепровождении в интернете
%	\item у наших пользователей появляется возможность монетизировать свое время, проведенное в интернете (данный вид заработка особенно актуален в странах с низким уровнем доходов и занятости населения. Молодые люди, проходящие обучение в средних или высших учебных заведениях, так или иначе использующие интернет в своих образовательных и развлекательных целях, теперь могут еще и получать за это деньги. Более того, подобный способ получение дохода как никогда актуален в текущий период мирового экономического кризиса, массового сокращения рабочих мест и увеличение доли малообеспеченного населения).
%\end{itemize}
%\bigbreak

%{\large Минусы:}
%\begin{itemize}
%	\item крупные рекламные площадки либо не одобряют такую схему разделения доходов, либо пока относятся к этому индифферентно
%\end{itemize}
%\bigbreak

%(большая аудитория VR приложения (в каких странах пользуются?) -> (гипотеза) ожидается рост аудитории браузера (желательно сроки и цифры, хотя бы примерно) в таких-то странах (список на основе статистики использования Browser / VR (так как можно перелить часть аудитории в браузер)) -> (гипотеза) в этих странах локальные рекламодатели, которые захотят получить доступ к нашей аудитории? (нужна ли она им? скорее всего да, если довольно большая) -> (гипотеза) используют нашу рекламную сеть (какие есть конкуренты? почему они должны выбрать именно нас?)
%\bigbreak

\bigbreak
{\large \textbf{Краткое описание структуры работы}}
\bigbreak

В первой главе рассмотрен вопрос…

В второй главе рассмотрен вопрос…

В третьей главе рассмотрен вопрос…

Заключение ...




%% Начало содержательной части.




\chapter{Обзор предметной области}


\startrelatedwork %% Так помечается начало обзора.

\begin{quotation}
  Описание того как на данный момент устроены рекламные платформы, термины и общая структура
\end{quotation}

Задачи и цели таргетированной рекламы
Как и любой другой канал коммуникации с пользователями, таргет преследует несколько целей: информирование юзеров о бренде или продукте, продажа товара, привлечение внимания, обучение потребителей.

Эксперты СММ выделяют несколько видов задач, которые решает таргет:

Сбор целевых посетителей, которые интересуются продуктом или услугой и готовы покупать.
Быстрое донесение информации о продукте, бренде, акциях до ЦА и побуждение посетителей перейти на источник для подробного ознакомления.
Совершение целевого действия на месте – покупка, заявка, подписка, регистрация и другое.
Принцип работы
В основе технологии таргетинговой рекламы лежит принцип получения наиболее полной информации о пользователе. Далее система обрабатывает массив данных, сегментируют посетителей по категориям в соответствии с метриками. В соцсетях этот принцип реализуется максимально просто, так как при регистрации пользователи добровольно вводят личные данные в поля анкеты.

Например, Вконтакте предлагает заполнить такие поля: пол, возраст, день рождения, любимые фильмы, семейное положение, интересы и другие. Вся информация потом используется в таргетированной рекламе и не только.

Объявления после модерации показываются целевой аудитории, согласно настройкам кампании.

	



Провести анализ предметной области на основе первоначальных требований: существуют ли готовые решения или отдельные компоненты.
В данной предметной области достаточно много различных терминов, поэтому 
определим их прежде чем сформулируем подробную цель работы.

\section{Используемые термины и понятия}\label{sec:terms}

%%

\textbf{Рекламная сеть (Ad Network)}
\\
An advertising buying network is an online platform that acts as an intermediary between advertisers and publishers. It collects the publisher's inventory and matches it with demand coming from advertisers. Ad networks are able to sort inventory into categories according to the specific audience segments.
\\
Целевая рекламная сеть (targeted ad network) работает точно. Наиболее продвинутый вид сетей, в связи с чем их также называют сетями 2.0. Такие сети специализируются на технологиях точного таргетирования по поведенческим или контекстным признакам, активно анализируют данные о пользователях в целях повышения стоимости инвентаря, который они продают
\\
Дополнительно описание берем отсюда https://smartyads.com/blog/create-your-own-ad-network/
\\
Примеры рекламных сетей: Facebook, Google AdMob, ...

%%

\textbf{Основные роли в рекламной сети}
\\
Рекламодатель (Advertiser)
\\
Рекламодатель как пользователь рекламной сети:
создает рекламные материалы различных форматов (ad format) = "типов инвентаря" (inventory type) рекламной сети
конфигурирует (таргетирует) аудиторию для данного рекламного материала
указывает белый / черный список возможных плейсментов (отдельные площадки или категории площадок, отдельные разделы конкретных площадок, …)
\\
Издатель (Publisher)
\\
Publisher как пользователь рекламной сети:
создает (регистрирует) плейсменты доступных "типов инвентаря" рекламной сети, указывает параметры площадки: тематику, тип и объем доступной ему аудитории и т.д.
указывает белый / черный список возможных рекламодателей / рекламных агенств (отдельные или по категориям)
\\
Advertisers don't know which websites serve their ads. Publishers don't know what companies buy their inventory.

%%

\textbf{DSP / SSP, Ad server}
\\
... тут описания понятий и схемы технического устройства

%%

\textbf{Таргетирование}
Решение о показе того или иного рекламного объявления принимается на основе следующей информации:
\begin{itemize}
	\item контекста плейсмента (inventory placement) для рекламы (Content and contextual targeting - \url{https://en.wikipedia.org/wiki/Contextual_advertising}),
	\item накопленных данных о пользователе (Behavioral targeting - \url{https://en.wikipedia.org/wiki/Targeted_advertising#Behavioural_targeting}): страна, соц-дем, устройства, категории интереса (подробнее это все расписано в TargetingOptions) (Amazon <-> IBM Watson category mapper)
\end{itemize}
\bigbreak

Targeting Options \url{https://adprofs.co/how-to-evaluate-demand-side-platform-companies/#targeting}

%%

\textbf{Трекинг}
Tracking data: Track / Impression / Click / Sell
\\
Основная воронка трекинга, open vs closed funnels

%%

\begin{quotation}
  Сайт = рекламная площадка (в нашем случае это мобильное приложение)
  \\
  OpenRate = fill rate

\end{quotation}

%%

\section{Основной бизнес-процесс}

стадия 0: поиск и интеграция исходных рекламодателей
\\
результат 0: исходные рекламодатели (Amazon, EBay, Aliexpress, ...)
\\
стадия 1: фильтрация товаров и обогащение сторонними данными (отзывы, локализация, …)
\\
результат 1: рекламные объявления UniAds или рекламные провайдеры: UniAds, Facebook, Google AdMob, …
\\
стадия 2: фильтрация и ранжирование уже созданных объявлений
\\
результат 2: показы объявлений
\\
стадия 3:
\\
результат 3: переходы (клики)
\\
стадия 4:
\\
результат 4: конверсии (продажи, установки приложения, …)
\\
стадия 5: выделение конверсий с максимальной возможной прибылью
\\
результат 5: прибыль
\\	

\section{Постановка задачи}


Для достижения поставленной цели необходимо решить основную задачу, которая заключается в разработке программного обеспечения, необходимого для полноценного функционирования рекламной сети.

Для удобства и простоты дальнейшего описания подзадач, под аббревиатурой ПО будем понимать программное обеспечение, необходимое для решение основной задачи.

Решение основной задачи согласно выбранной (каскадной) модели разработки состоит из следующих стадий:
\begin{itemize}
	\item сформировать требования к ПО
	\item спроектировать архитектуру ПО
	\item реализовать ПО на основе требований и модели архитектуры
	\item протестировать итоговое ПО
	\item внедрить ПО
\end{itemize}
\bigbreak


\section{Требования}\label{sec:requirements}


\textbf{Общие базовые (Бизнес-требования)}
\begin{itemize}
	\item Возможность показывать товары с Amazon в виде рекламных объявлений
	\item Компания Fulldive единственный Publisher и единственный Advertiser (в будущем могут быть и другие рекламодатели)
	\item Модели ценообразования CPC и CPM (другие не поддерживаем)
	\item возможность создания и получения единичных объявлений
	\item возможность получения списка рекламных объявлений для показа на одной странице
	\item глобальный Postback для учета конверсий: покупок, установок приложения и т.д. (нужен только для модели CPA: CPI, CPS, …)
	\item трекинг данных статистики показов, переходов и пр. каждого рекламного объявления
	\item сбор данных по данным трекинга
\end{itemize}

\textbf{Требования к показу единичных объявлений:}
\begin{itemize}
	\item возможность задания категории объявления
	\item возможность задания списка стран, в которых будет показано объявление
\end{itemize}

\textbf{Требования к показу списка объявлений:}
\begin{itemize}
	\item консистентность списка для пользователя / конкретного девайса пользователя
	\item возможность фильтрации провайдеров / товаров?
	\item возможность учета для каждого товара из списка: показов и кликов
\end{itemize}

\textbf{Требования к форматам рекламных объявлений}
\begin{itemize}
	\item поддержка нативного формата (изображения с различными разрешениями + текст)
	\item поддержка баннерного формата (HTML разметка)
	\item поддержка видеоформата
\end{itemize}

\textbf{Требования к аналитике:}
\begin{itemize}
	\item возможность просмотра статистики каждого рекламного объявления в формате (AdId, Показы, Клики, CTR, тут колонки таблицы) с фильтрацией по … и группировкой по …
	\item возможность просмотра параметров эффективности (OR, CTR) стратегий таргетинга, рассчитанных по данным трекинга, с фильтрацией по интервалу дат и группировкой по идентификатору стратегии
\end{itemize}

\textbf{Технические требования:} количественные параметры (количество показов, кликов в секунду / сутки, максимальное количество ошибок за сутки / месяц)

\textbf{Общие сценарии использования}
\begin{itemize}
	\item Издатель запрашивает одно рекламное объявление
	\item Издатель запрашивает список рекламных объявлений
\end{itemize}

\textbf{Сценарии использования накопленных данных трекинга (ответ на вопрос: для чего мы собираем эти данные?):}
\begin{itemize}
	\item (ручное использование) расчет (real-time?) статистики по рекламным объявлениям: Impressions, Clicks, Conversions? (в основном для сторонних рекламодателей), CTR, CR для отображение в панели рекламодателя
	\item (автоматическое использование) использование CTR рекламного объявления в стратегиях таргетирования (Global Interesting Strategy, Interesting in Category Strategy, …)
	\item (ручное использование) оптимизация бизнес процесса на основе данных (Data-driven decision making)
	\item  (автоматическое использование) автоматический выбор стратегии таргетирования на основе статистики стратегий
\end{itemize}


\section{Обзор существующих решений}\label{sec:requirements}

\finishrelatedwork %% Так помечается конец обзора.


\chapterconclusion


\chapter{Описание предложенного решения}

В данной главе описываются основные компоненты рекламной сети, клиент-серверное взаимодействие, ...

Исходя из требований и сценариев использования приведенных в \ref{sec:requirements} было принято решение выбрать трёхуровневую клиент-серверную архитектуру, в которой:
\begin{itemize}
	\item Cервер (Ad Server - \url{https://en.wikipedia.org/wiki/Ad_serving}) отвечает за управление рекламным инвентарем и рекламными объявлениями, сбор статистики показов, переходов и конверсий. Состоит из связующего слоя (приложения) и слоя данных (базы данных).
	\item Клиент отвечает за корректное отображение полученных от сервера рекламных материалов.
\end{itemize}

\section{Клиент-серверное взаимодействие}

Взаимодействие клиента и сервера происходит по заранее установленному контракту, реализованному поверх HTTP и AMQP протоколов и представляет из себя несколько сценариев, которые будут подробнее описаны далее.

\subsection{Получение рекламного объявления}
Получение данных пользователя. 
Выбор стратегии таргетинга. 
Получение списка объявлений по выбранной стратегии.

\subsection{Получение списка рекламных объявлений}
Получение данных пользователя и текущей сессии (у каждой сессии свой уникальный ключ). 
Выбор стратегии таргетинга. 
Получение списка объявлений по выбранной стратегии.

\subsection{Трекинг показов и переходов}
...

\subsection{Получение статистики по стратегиям и рекламных объявлениям}
...

\section{Таргетирование}


Для таргетирования используются 4 стратегии:
\begin{enumerate}
	\item случайные объявления
	\item объявления с глобально высоким показателем CTR
	\item объявления с высоким показателем CTR в категориях интереса данного пользователя, рассчитанных на основе просмотренных страниц в интернете
	\item объявления с высоким показателем CTR в категориях интереса данного пользователя, рассчитанных на основе кликов по объявлениям внутри приложения
\end{enumerate}

Каждая из приведенных стратегия таргетирования назначает веса рекламным объявлениям \url{https://en.wikipedia.org/wiki/Creative_sequencing}, чтобы управлять их частотой показа. Также применяется рандомизация для равномерного показа объявлений внутри своей группы (подтвержденные, отклоненные, тестируемые)

Выбор стратегии на основе данных, представленных в запросе:

Если рекламное объявление должно зависеть от контекста, в котором оно будет показано выбирается одна из стратегий, которая таргетирует на основании параметров переданного контекста. Иначе выбирается стратегия, которая таргетирует на основании накопленных знаниях о категориях интереса текущего пользователя.

\section{Актуализация рекламных материалов}

Рекламодатель создает рекламные материалы на основе товаров из Amazon, сохраняет их в базе данных и обновляет список актуальных рекламных материалов, доступных для получения клиентами.

(Схема)


\section{Актуализация рекламного инвентаря}

Издатель запрашивает актуальный рекламный инвентарь на сервере конфигурации и сохраняет его в базе данных.

(Схема)

\chapterconclusion

Вывод в конце второй главы

\chapter{Техническая реализация и внедрение предложенной архитектуры}

Решено было реализовать ПО в виде JSON API

Здесь описывается итоговая архитектура рекламной сети (инфраструктура, количество серверов, тип масштабирования, базы данных, кэширование...

\section{Техническая подзадача 1}
выбор оптимальной стратегии таргетирования (A/B тестирование стратегий)
как решать 
почему именно так решать

\section{Техническая подзадача 2}
StrategyState resolving (N.B. sessionKey по идее может быть произвольной строкой предоставляемой Publisher'ом)

\section{Техническая подзадача 3}
StrategyState threads (обновление состояний потока по расписанию (а также при запросе?))

\section{Техническая подзадача 4}
таргетирование = ранжирование (относительные величины - порядковые номера)/ скоринг (абсолютные величины)+ эвристики фильтрации + небольшая рандомизация (в будущем можно сделать на основе TargetingOptions (для этого можно использовать сторонее решение? какое?)

\section{Техническая подзадача 5}
трекинговая закрытая воронка (Track / Impression / Click / Sell)

\section{Техническая подзадача 6} актуальность кэша рекламных объявлений


\chapterconclusion

Вывод в конце третьей главы

%% Макрос для заключения. Совместим со старым стилевиком.
\startconclusionpage

Дальнейшее развитие (стратегические направления / цели) (исходя из них можно уточнить требования к реализации платформ
(все дальнейшие разработки и улучшения также направлены на увеличение прибыли (согласно основному бизнес-процессу итоговая цель - прибыль))
№1 Развитие компании в качестве своего собственного рекламодателя (growth as Advertiser / Ad provider)
Как это сделать?
Идея №1: увеличить количество исходных рекламодателей - интегрировать E-commerce платформы Ebay, Aliexpress, Wildberries, …
Идея №2: подбирать множество товаров таким образом, чтобы итоговое число покупок было максимально (слишком мало товаров - неудовлетворенный спрос, слишком много товаров - проблема выбора и рассеивание внимания)
Идея №3: тиражирование товаров с максимально высокой вероятностью покупки для конкретного пользователя, например, товары из интересующих пользователя категорий (персонализация показываемой рекламы)
№2 Оптимизация количества и качества плейсментов внтури приложений компании (growth as Publisher)
№3 Привлечение сторонних рекламодателей (добавление RTB / Bidders)

\printmainbibliography

%% После этой команды chapter будет генерировать приложения, нумерованные русскими буквами.
%% \startappendices из старого стилевика будет делать то же самое

\appendix

%\chapter{Пример приложения}\label{sec:app:1}
                

\end{document}